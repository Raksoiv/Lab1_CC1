\documentclass[10pt|letterpaper]{article}
\usepackage{color}
\usepackage[utf8]{inputenc} 
\usepackage{graphicx}
\usepackage{float}
\usepackage{enumerate}

\begin{document}
\begin{center}{
\huge\textbf{Laboratorio 1 \\ Computación Científica I}} \\
\bigskip
\textit{I) Representación Punto Flotante \\ II) Pérdida de significancia}

\vspace*{5.0 in}
\small{INTEGRANTES:} \\
\small{Oscar Rencoret oscar.rencoret@alumnos.usm.cl 2011735352-8} \\
\small{Ximena Rojas ximena.rojas@alumnos.usm.cl 201173568-4}
\end{center}

\newpage

\section{Introducción}
Este informe tiene como objetivo resolver y analizar los ejercicios propuestos que ponen en evidencia las limitaciones de realizar calculos aritmeticos en un computador y que se deben tener en cuenta al realizar operaciones aritmeticas.

\section{Pequeña descripción de los experimentos}
\begin{itemize}
\item Se pone a prueba la precisión del ordenador, realizando operaciones aritmeticas en ordenes de magnitud alrededor del $\epsilon_{match}$ 
\item Se representan gráfcamente los numeros decimales en notación de punto flotante con la finalidad de evidenciar la perdida de precición.
\end{itemize}

\section{Parte I: Representación Punto Flotante}

\subsection{Pregunta 1}

\begin{enumerate}[a)]
\item Calcule las distancias aritméticamente.\\
Aritméticamente las sumas dan como resultado:
\begin{enumerate}[i.]
\item $((1 + 2^{-52}) - 1) + 2^{-54} + 1 = 2^{-52} + 2^{-54} + 1\\
2^{-52} + 2^{-54} + 1 = 2^{2} * 2^{-54} + 2^{-54} + 1\\
2^{2} * 2^{-54} + 2^{-54} + 1 = 2^{-54}( 2^{2} + 1) + 1\\
2^{-54}( 2^{2} + 1) + 1 = 2^{-54} * 5 + 1$
$$5 * 2^{-54} + 1 = 1.000000000000000277555756$$
\newline 
\newline
$((1 + 2^{-54}) - 1) + 2^{-52} + 1 = 2^{-54} + 2^{-52} + 1\\
2^{-54} + 2^{-52} + 1 = 2^{-54} + 2^{2} * 2^{-54} + 1 \\
2^{-54} + 2^{2} * 2^{-54} + 1 = 2^{-54}( 1 + 2^{2}) + 1\\
2^{-54}( 1 + 2^{2}) + 1 = 2^{-54} * 5 + 1$
$$5 * 2^{-54} + 1 = 1.000000000000000277555756$$
\item $(5 - 4) + 2^{-52} = 1 + 2^{-52}$ 
$$1 + 2^{-52} = 1.000000000000000222044605$$
\newline 
\newline
$5 - (4 - 2^{-52}) = 5 - 4 - 2^{-52}\\
5 - 4 - 2^{-52} = 1 - 2^{-52}$
$$1 - 2^{-52} = 0.999999999999999777955395$$
\item $(2^{53} + (-2^{53})) + (1 + 0.5 + 0.25) = (2^{53} - 2^{53}) + 1.75$
$$(2^{53} - 2^{53}) + 1.75 = 1.75$$
\newline 
\newline
$(2^{53} + (1 + 0.5 + 0.25)) - 2^{53} = 2^{53} + 1.75 -  2^{53}$ 
$$2^{53} + 1.75 -  2^{53} = 1.75$$
\end{enumerate}

\item Con la ayuda de \textit{Python} o \textit{Matlab}, calcule nuevamente las distancias, esta vez computacionalmente, analizando cada caso del calculo (desarrollo de los paréntisis) y como influyen en el resultado\\
Lo resultados son:
\begin{verbatim}
i.1
((1 + 2**(-52)) - 1) + 2**(-54) + 1
((1.0000000000000002) - 1) + 2**(-54) + 1
(2.220446049250313e-16) + 2**(-54) + 1
2.7755575615628914e-16 + 1
1.0000000000000002
i.2
((1 + 2**(-54)) - 1) + 2**(-52) + 1
((1.0) - 1) + 2**(-52) + 1
(0.0) + 2**(-52) + 1
2.220446049250313e-16 + 1
1.0000000000000002

ii.1
(5 - 4) + 2**(-52)
1 + 2**(-52)
1.0000000000000002
ii.2
5 - (4 - 2**(-52))
5 - 4.0
1.0

iii.1
(2**(53) + (-2**(53)) + (1 + .5 + .25)
0 + 1.75
1.75
iii.2
(2**(53) + (1 + .5 + .25)) - 2**(53)
9007199254740994.0 - 2**(53)
2.0
\end{verbatim}
El código utilizado para resolver las sumas se encuentra en la carpeta "Códigos"

\item Defina que es $\epsilon_{match}$\\
$\epsilon_{match}$ es la distancia que existe entre el número 1 y el menor número siguiente en representación punto flotante que es mayor a 1.
En representación con doble precisión de punto flotante, corresponde a $1 + 2^{-52}$, por lo que la distancia corresponde  a $\epsilon_{mach} = 2^{-52}$.

\item Analice y concluya: ¿Existen diferencias entre los resultados aritméticos y computacionales?.
De ser así, explique porque ocurren estas diferencias\\

\item ¿Qué distancias son representables en el sistema?

\end{enumerate}

\subsection{Pregunta 2}

\subsection{Pregunta 3}

\section{Parte II: Pérdida de Significancia}

\subsection{Pregunta 1}

\section{Conclusión}

\section{Referencias}

\end{document}
